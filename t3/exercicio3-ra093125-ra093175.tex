\documentclass[a4paper,10pt]{article}
\usepackage[utf8]{inputenc}
\usepackage[top=2cm, left = 2cm , right=2cm , bottom=2cm]{geometry}
\usepackage{amsmath}
\usepackage{amsfonts}
\usepackage{amssymb}
\usepackage{graphicx}
\usepackage{float}
\usepackage{subcaption}
\usepackage[brazil]{babel}
%\pagestyle{plain}
\usepackage{listings}
\usepackage{color}
\usepackage{hyperref}

\definecolor{dkgreen}{rgb}{0,0.6,0}
\definecolor{gray}{rgb}{0.5,0.5,0.5}
\definecolor{mauve}{rgb}{0.58,0,0.82}

\lstset{frame=tb,
  language=bash,
  aboveskip=3mm,
  belowskip=3mm,
  showstringspaces=false,
  columns=flexible,
  basicstyle={\small\ttfamily},
  numbers=none,
  numberstyle=\tiny\color{gray},
  keywordstyle=\color{blue},
  commentstyle=\color{dkgreen},
  stringstyle=\color{mauve},
  breaklines=true,
  breakatwhitespace=true,
  tabsize=3,
  literate={á}{{\'a}}1
           {ç}{{\c{c}}}1
           {ü}{{\"u}}1
}

\lstset{
}

\hypersetup{
    colorlinks=true,
    linkcolor=blue,
    filecolor=magenta,      
    urlcolor=cyan,
}
 
\urlstyle{same}

\begin{document}
%\twocolumn


\title{MC833 A - Programação de redes de computadores\\
Relatório - Tarefa 03}

\author {   093125 - Tiago Martinho de Barros - \textit{tiago.ec09@gmail.com}\\
            093175 - Victor Fernando Pompeo Barbosa - \textit{victorfpb@gmail.com}}

%\date{}

\maketitle

\centerline{Prof. Paulo Licio de Geus}
\centerline{IC -- UNICAMP}

\vspace{2cm}
\tableofcontents
    
%%%%%%%%%%%%%%%%%%%%%%%%%%%%%%%%%%%%%%%%%%%%%%%%%%%%%%%%%%%%%%%%%%%%%%%%%%%%%
\newpage
\section{Introdução}
\hspace{14pt}

    Nesta tarefa estudaremos a abstração de {\tt socket} na linguagem C e as chamadas de sistema associadas. A importância dessas chamadas reside na sua utilidade para construção de aplicações de rede.

%%%%%%%%%%%%%%%%%%%%%%%%%%%%%%%%%%%%%%%%%%%%%%%%%%%%%%%%%%%%%%%%%%%%%%%%%%%%%

\section{Desenvolvimento}

Este exercício consiste basicamente no estudo de funções usadas na programação de softwares que se utilizam de rede para transmitir e receber mensagens. E também na compreensão do programa \textit{Simplex-Talk}, que é um programa simples de comunicação via redes, bem como em modificações dele.

\section{Questão 1}
Nesta questão, explicaremos o funcionamento de duas funções: {\tt accept} e {\tt htons}.
\begin{enumerate}
\item {\tt accept}\\
    Esta função está definida em $\langle sys/socket.h \rangle$ e sua assinatura é:
    \begin{lstlisting}
    int accept(int socket, struct sockaddr *restrict address,
        socklen_t *restrict address_len);
    \end{lstlisting}
    O que ela faz é pegar a primeira conexão da lista de conexões pendentes, criar um novo \textit{socket} com o mesmo protocolo e família de endereço que o \textit{socket} passado como parâmetro, e também alocar um novo descritor de arquivo para o \textit{socket} criado.
    
    Os parâmetros são:
    \begin{enumerate}
        \item \textit{socket}\\
        Um \textit{socket} criado com a função {\tt socket()}, que foi associado a um endereço com a função {\tt bind()} e que foi usado na função {\tt listen()}.
        \item \textit{address}\\
        Apontador para uma estrutura \textbf{sockaddr} em que o endereço do \textit{socket} usado na conexão deve ser retornado. Pode ser NULL.
        \item \textit{address\_len}\\
        Apontador para um objeto \textbf{socklen\_t} que, na entrada, especifica o tamanho da estrutura \textbf{sockaddr} especificada e, na saída, especifica o tamanho do endereço armazenado. Pode ser NULL se \textit{address} for NULL.
    \end{enumerate}
    
    O valor de retorno é um descritor de arquivo não negativo se a chamada foi completada com sucesso. Caso contrário, -1 é retornado.
\item {\tt htons}\\
    Esta função está definida em $\langle arpa/inet.h \rangle$ (ou em $\langle netinet/in.h \rangle$) e sua assinatura é:
    \begin{lstlisting}
    uint16_t htons(uint16_t hostshort);
    \end{lstlisting}
    Ela recebe um \textbf{unsigned short integer} (inteiro curto, sem sinal, de 16 bits), encodado na arquitetura de ordem de bytes do cliente (no caso do i386, a codificação é \textit{Little Endian}), e o converte para a arquitetura de ordem de bytes da rede (na Internet, a codificação é \textit{Big Endian}). Na arquitetura \textit{Big Endian}, o byte mais significativo de uma palavra é armazenado em certo endereço de memória e os bytes subsequentes são guardados nos próximos endereços, de maior valor. A arquitetura \textit{Little Endian}, por sua vez, armazena o byte menos significativo no endereço de memória de menor valor e o byte mais significativo no endereço de memória de maior valor.
\end{enumerate}

\section{Questão 2}
    Os programas fornecidos no site da disciplina foram compilados e executados numa mesma máquina. A entrada/saída do servidor pode ser verificada a seguir.
    
    \begin{lstlisting}
        niko@ubuntu:~/Desktop/mc833/t3/code$ ./server
        Testing
        client2servermessage
        Trying whitespaces now
        Isso funciona com sinais gráficos?
        E se colocarmos c cedilha? ç
    
    \end{lstlisting}
    
    A entrada/saída do cliente, por sua vez, pode ser verificada a seguir.
    
    \begin{lstlisting}
        niko@ubuntu:~/Desktop/mc833/t3/code$ ./client localhost
        Testing
        client2servermessage   
        Trying whitespaces now
        Isso funciona com sinais gráficos?
        E se colocarmos c cedilha? ç
    
    \end{lstlisting}
    
    Como pode ser observado, a comunicação ocorreu perfeitamente, transmitindo inclusive sinais gráficos e cê-cedilha.

\section{Questão 3}
    A primeira ferramenta utilizada para comprovação de que os códigos estão realizando uma comunicação via rede é o comando {\tt tcpdump}. Invocando o comando com instruções explícitas para que escutasse a interface de loopback, pudemos identificar os pacotes transmitidos por meio dos códigos executados. A entrada/saída do comando {\tt tcpdump} está exibida a seguir.
    
    \begin{lstlisting}
    niko@ubuntu:~/Desktop/mc833/t3/code$ sudo tcpdump -i lo
        tcpdump: verbose output suppressed, use -v or -vv for full protocol decode
        listening on lo, link-type EN10MB (Ethernet), capture size 65535 bytes
        16:16:02.844376 IP localhost.52844 > localhost.31472: Flags [P.], seq 2140592837:2140592869, ack 1074105590, win 342, options [nop,nop,TS val 21866426 ecr 21863306], length 32
        16:16:02.844402 IP localhost.31472 > localhost.52844: Flags [.], ack 32, win 342, options [nop,nop,TS val 21866426 ecr 21866426], length 0
        16:16:06.592760 IP localhost.50765 > ubuntu.domain: 53303+ A? www.papeeria.com. (34)
        16:16:06.592928 IP localhost.50765 > ubuntu.domain: 39377+ AAAA? www.papeeria.com. (34)
        16:16:06.611305 IP ubuntu.domain > localhost.50765: 53303 2/4/8 CNAME papeeria.com., A 104.155.84.196 (358)
        16:16:06.624588 IP ubuntu.domain > localhost.50765: 39377 1/1/0 CNAME papeeria.com. (138)
        16:16:07.257027 IP localhost.52844 > localhost.31472: Flags [P.], seq 32:72, ack 1, win 342, options [nop,nop,TS val 21867529 ecr 21866426], length 40
        16:16:07.257054 IP localhost.31472 > localhost.52844: Flags [.], ack 72, win 342, options [nop,nop,TS val 21867529 ecr 21867529], length 0
        16:16:11.630504 IP localhost.52844 > localhost.31472: Flags [P.], seq 72:133, ack 1, win 342, options [nop,nop,TS val 21868622 ecr 21867529], length 61
        16:16:11.630529 IP localhost.31472 > localhost.52844: Flags [.], ack 133, win 342, options [nop,nop,TS val 21868622 ecr 21868622], length 0
        16:16:12.992878 IP localhost.40878 > ubuntu.domain: 31618+ A? ads.papeeria.com. (34)
        16:16:12.993188 IP localhost.40878 > ubuntu.domain: 45969+ AAAA? ads.papeeria.com. (34)
        16:16:12.997273 IP localhost.41561 > ubuntu.domain: 17325+ A? www.papeeria.com. (34)
        16:16:12.997298 IP localhost.41561 > ubuntu.domain: 3480+ AAAA? www.papeeria.com. (34)
        16:16:13.015900 IP ubuntu.domain > localhost.41561: 3480 1/1/0 CNAME papeeria.com. (138)
        16:16:13.016243 IP ubuntu.domain > localhost.40878: 45969 0/1/0 (124)
        16:16:13.017286 IP ubuntu.domain > localhost.41561: 17325 2/4/8 CNAME papeeria.com., A 104.155.84.196 (358)
        16:16:13.144504 IP ubuntu.domain > localhost.40878: 31618 1/4/8 A 104.155.12.58 (344)

    \end{lstlisting}
    
    A entrada/saída do programa que funciona como servidor pode ser verificada a seguir.
    
    \begin{lstlisting}
        niko@ubuntu:~/Desktop/mc833/t3/code$ ./server
        This message has 30 characters
        And now we're writing some other stuff
        Just to see if tcpdump can correctly identify these packets
    
    \end{lstlisting}
    
    E, por fim, a seguir podem ser encontradas a entrada/saída do programa que funciona como cliente.
    
    \begin{lstlisting}
        niko@ubuntu:~/Desktop/mc833/t3/code$ ./client localhost
        This message has 30 characters
        And now we're writing some other stuff
        Just to see if tcpdump can correctly identify these packets

    \end{lstlisting}

    As seções da saída do {\tt tcpdump} que permitem identificar que a comunicação se dá via rede estão exibidas a seguir.
    
    \begin{lstlisting}
        16:16:02.844376 IP localhost.52844 > localhost.31472: Flags [P.], seq 2140592837:2140592869, ack 1074105590, win 342, options [nop,nop,TS val 21866426 ecr 21863306], length 32
        16:16:02.844402 IP localhost.31472 > localhost.52844: Flags [.], ack 32, win 342, options [nop,nop,TS val 21866426 ecr 21866426], length 0
        
    \end{lstlisting}

    \begin{lstlisting}
        16:16:07.257027 IP localhost.52844 > localhost.31472: Flags [P.], seq 32:72, ack 1, win 342, options [nop,nop,TS val 21867529 ecr 21866426], length 40
        16:16:07.257054 IP localhost.31472 > localhost.52844: Flags [.], ack 72, win 342, options [nop,nop,TS val 21867529 ecr 21867529], length 0
    
    \end{lstlisting}
    
    \begin{lstlisting}
            16:16:11.630504 IP localhost.52844 > localhost.31472: Flags [P.], seq 72:133, ack 1, win 342, options [nop,nop,TS val 21868622 ecr 21867529], length 61
            16:16:11.630529 IP localhost.31472 > localhost.52844: Flags [.], ack 133, win 342, options [nop,nop,TS val 21868622 ecr 21868622], length 0
    \end{lstlisting}

    Como pode ser verificado, a comunicação se dá entre as mesmas portas do {\tt localhost}, além de a contagem de bytes de cada uma das mensagens equivaler exatamente ao tamanho das mensagens transmitidas utilizando o par de programas.\\
    
    Outra ferramenta que podemos utilizar para comprovar que os códigos estão realizando uma comunicação via rede é o {\tt netstat}, que fornece informações sobre as conexões ativas. Executando o servidor e o cliente numa mesma máquina, temos:
    
    \begin{lstlisting}
        bash-4.3$ netstat -tuW
        Active Internet connections (w/o servers)
        Proto Recv-Q Send-Q Local Address           Foreign Address         State      
        tcp        0      0 taz.lab.ic.unicamp.br:56978 40.86.88.211:https      ESTABLISHED
        tcp        0      0 taz.lab.ic.unicamp.br:36146 196.84.155.104.bc.googleusercontent.com:https ESTABLISHED
        tcp        0      0 taz.lab.ic.unicamp.br:40436 cb-in-f189.1e100.net:https ESTABLISHED
        tcp        0      0 taz.lab.ic.unicamp.br:44660 msnbot-65-52-108-76.search.msn.com:https ESTABLISHED
        tcp        0      0 localhost.localdomain:56562 localhost.localdomain:31472 ESTABLISHED
        tcp        0      0 taz.lab.ic.unicamp.br:40378 58.12.155.104.bc.googleusercontent.com:https ESTABLISHED
        tcp        0      0 taz.lab.ic.unicamp.br:36068 74.125.107.203:https    ESTABLISHED
        tcp        0      0 taz.lab.ic.unicamp.br:35574 196.84.155.104.bc.googleusercontent.com:https ESTABLISHED
        tcp        0      0 taz.lab.ic.unicamp.br:33668 rio01s16-in-f5.1e100.net:https ESTABLISHED
        tcp        0      0 localhost.localdomain:31472 localhost.localdomain:56562 ESTABLISHED
        tcp        0      0 taz.lab.ic.unicamp.br:wpgs franjinha.lab.ic.unicamp.br:nfs ESTABLISHED
        tcp        0      0 taz.lab.ic.unicamp.br:33700 rio01s15-in-f1.1e100.net:https ESTABLISHED
        tcp        0      0 taz.lab.ic.unicamp.br:35128 edge-star-shv-04-prn2.facebook.com:https ESTABLISHED
        tcp        0      0 taz.lab.ic.unicamp.br:36206 gru06s26-in-f14.1e100.net:https ESTABLISHED
        tcp        0      0 taz.lab.ic.unicamp.br:59990 franjinha.lab.ic.unicamp.br:sunrpc TIME_WAIT  
        tcp        0      0 taz.lab.ic.unicamp.br:48726 cb-in-f93.1e100.net:https ESTABLISHED
        tcp        0      0 taz.lab.ic.unicamp.br:49292 mingau.lab.ic.unicamp.br:ldap ESTABLISHED
        tcp        0      0 taz.lab.ic.unicamp.br:34626 72.21.81.200:https      ESTABLISHED
        tcp        0      0 taz.lab.ic.unicamp.br:34024 cb-in-f138.1e100.net:https ESTABLISHED
        tcp        0      0 taz.lab.ic.unicamp.br:928 franjinha.lab.ic.unicamp.br:47518 ESTABLISHED

    \end{lstlisting}
    
    Duas linhas mostram a conexão via rede na máquina:
    
    \begin{lstlisting}
        tcp        0      0 localhost.localdomain:56562 localhost.localdomain:31472 ESTABLISHED
        tcp        0      0 localhost.localdomain:31472 localhost.localdomain:56562 ESTABLISHED
    \end{lstlisting}
    
    Como sabemos, pelo código do servidor e do cliente, o servidor escuta na porta 31472, que aparece nas saídas acima. Também notamos que a porta efêmera usada pelo cliente neste caso foi 56562.

\section{Questão 4}
    Sim, o programa {\tt telnet} pode ser usado no lugar do binário do {\tt client.c}, pois podemos usar o programa {\tt telnet} para conectar a um determinado host numa determinada porta, como abaixo:
    
    \begin{itemize}
        \item Saída da execução do {\tt telnet}
            \begin{lstlisting}
    bash-4.3$ telnet localhost 31472
    Trying 127.0.0.1...
    Connected to localhost.
    Escape character is '^]'.
    testando telnet
        
            \end{lstlisting}
    
        \item Saída do servidor
            \begin{lstlisting}
    bash-4.3$ ./server 
    testando telnet
    
            \end{lstlisting}
    \end{itemize}

\section{Questão 5}
    Os programas modificados são incluídos abaixo:
    
    \begin{itemize}
        \item {\tt server.c}
            \begin{lstlisting}
    #include <stdlib.h>
    #include <stdio.h>
    #include <string.h>
    #include <sys/types.h>
    #include <sys/socket.h>
    #include <netinet/in.h>
    #include <netdb.h>
    #include <unistd.h>
    
    #define SERVER_PORT 31472
    #define MAX_PENDING 5
    #define MAX_LINE 256
    
    int main()
    {
    	struct sockaddr_in sin;
    	char buf[MAX_LINE];
    	int len;
    	int s, new_s;
    
    	/* build address data structure */
    	bzero((char *)&sin, sizeof(sin));
    	sin.sin_family = AF_INET;
    	sin.sin_addr.s_addr = INADDR_ANY;
    	sin.sin_port = htons(SERVER_PORT);
    
    	/* setup passive open */
    	if ((s = socket(PF_INET, SOCK_STREAM, 0)) < 0) {
    		perror("simplex-talk: socket");
    		exit(1);
    	}
    	if ((bind(s, (struct sockaddr *)&sin, sizeof(sin))) < 0) {
    		perror("simplex-talk: bind");
    		exit(1);
    	}
    	listen(s, MAX_PENDING);
    
    	/* wait for connection, then receive and print text */
    	while(1) {
    		if ((new_s = accept(s, (struct sockaddr *)&sin, &len)) < 0) {
    			perror("simplex-talk: accept");
    			exit(1);
    		}
    		while ((len = recv(new_s, buf, sizeof(buf), 0))) {
    			fputs(buf, stdout);
                if((send(new_s, buf, strlen(buf)+1, 0)) < 0) {
                    perror("simplex-talk: send");
                    exit(1);
                }
            }
    		close(new_s);
    	}
    }
            \end{lstlisting}
        
        \item {\tt client.c}
            \begin{lstlisting}
    #include <stdlib.h>
    #include <stdio.h>
    #include <string.h>
    #include <sys/types.h>
    #include <sys/socket.h>
    #include <netinet/in.h>
    #include <netdb.h>
    #include <unistd.h>
    
    #define SERVER_PORT 31472
    #define MAX_LINE 256
    
    int main(int argc, char * argv[])
    {
    	struct hostent *hp;
    	struct sockaddr_in sin;
    	char *host;
    	char buf[MAX_LINE];
    	int s;
    	int len;
    	if (argc==2) {
    		host = argv[1];
    	}
    	else {
    		fprintf(stderr, "usage: ./client host\n");
    	exit(1);
    }
    
    	/* translate host name into peer's IP address */
    	hp = gethostbyname(host);
    	if (!hp) {
    		fprintf(stderr, "simplex-talk: unknown host: %s\n", host);
    		exit(1);
    	}
    
    	/* build address data structure */
    	bzero((char *)&sin, sizeof(sin));
    	sin.sin_family = AF_INET;
    	bcopy(hp->h_addr, (char *)&sin.sin_addr, hp->h_length);
    	sin.sin_port = htons(SERVER_PORT);
    
    	/* active open */
    	if ((s = socket(PF_INET, SOCK_STREAM, 0)) < 0) {
    		perror("simplex-talk: socket");
    		exit(1);
    	}
    	if (connect(s, (struct sockaddr *)&sin, sizeof(sin)) < 0) {
    		perror("simplex-talk: connect");
    		close(s);
    		exit(1);
    	}
    
    	/* main loop: get and send lines of text */
    	while (fgets(buf, sizeof(buf), stdin)) {
    		buf[MAX_LINE-1] = '\0';
    		len = strlen(buf) + 1;
    		send(s, buf, len, 0);
            recv(s, buf, sizeof(buf), 0);
            fputs(buf, stdout);
            fputs("\n", stdout);
    	}
    }
            \end{lstlisting}
    \end{itemize}

\end{document}

