\documentclass[a4paper,10pt]{article}
\usepackage[utf8]{inputenc}
\usepackage[top=2cm, left = 2cm , right=2cm , bottom=2cm]{geometry}
\usepackage{amsmath}
\usepackage{amsfonts}
\usepackage{amssymb}
\usepackage{graphicx}
\usepackage{float}
\usepackage{subcaption}
\usepackage[brazil]{babel}
%\pagestyle{plain}
\usepackage{listings}
\usepackage{color}
\usepackage{hyperref}

\definecolor{dkgreen}{rgb}{0,0.6,0}
\definecolor{gray}{rgb}{0.5,0.5,0.5}
\definecolor{mauve}{rgb}{0.58,0,0.82}

\lstset{frame=tb,
  language=bash,
  aboveskip=3mm,
  belowskip=3mm,
  showstringspaces=false,
  columns=flexible,
  basicstyle={\small\ttfamily},
  numbers=none,
  numberstyle=\tiny\color{gray},
  keywordstyle=\color{blue},
  commentstyle=\color{dkgreen},
  stringstyle=\color{mauve},
  breaklines=true,
  breakatwhitespace=true,
  tabsize=3
}

\hypersetup{
    colorlinks=true,
    linkcolor=blue,
    filecolor=magenta,      
    urlcolor=cyan,
}
 
\urlstyle{same}

\begin{document}
%\twocolumn


\title{MC833 A - Programação de redes de computadores\\
Relatório - Tarefa 03}

\author {   093125 - Tiago Martinho de Barros - \textit{tiago.ec09@gmail.com}\\
            093175 - Victor Fernando Pompeo Barbosa - \textit{victorfpb@gmail.com}}

%\date{}

\maketitle

\centerline{Prof. Paulo Licio de Geus}
\centerline{IC -- UNICAMP}

\vspace{2cm}
\tableofcontents
    
%%%%%%%%%%%%%%%%%%%%%%%%%%%%%%%%%%%%%%%%%%%%%%%%%%%%%%%%%%%%%%%%%%%%%%%%%%%%%
\newpage
\section{Introdução}
\hspace{14pt}

    Nesta tarefa estudaremos a abstração de {\tt socket} na linguagem C e as chamadas de sistema associadas. A importância dessas chamadas reside na sua utilidade para construção de aplicações de rede.

%%%%%%%%%%%%%%%%%%%%%%%%%%%%%%%%%%%%%%%%%%%%%%%%%%%%%%%%%%%%%%%%%%%%%%%%%%%%%

\section{Desenvolvimento}

O exercício se baseia em um cenário no qual um programa transmite um arquivo da máquina \textit{willow} para a máquina \textit{maple} sobre uma conexão TCP. A ferramenta {\tt tcpdump} foi executada no transmissor para registrar os pacotes enviados e os pacotes de reconhecimento recebidos.\\

O arquivo {\tt tcpdump.dat} contém o log de todos os pacotes TCP do cenário descrito e foi obtido a partir de um link fornecido na página da disciplina. O primeiro passo envolveu a conversão do arquivo {\tt tcpdump.dat} para o formato texto, por meio do comando {\tt tcpdump -r tcpdump.dat > outfile.txt}. \\

\section{Questão 1}
Nesta questão, explicaremos o funcionamento de duas funções: {\tt accept} e {\tt htons}.
\begin{enumerate}
\item {\tt accept}\\
    Esta função está definida em <sys/socket.h> e sua assinatura é:
    \begin{lstlisting}
    int accept(int socket, struct sockaddr *restrict address,
        socklen_t *restrict address_len);
    \end{lstlisting}
    O que ela faz é pegar a primeira conexão da lista de conexões pendentes, criar um novo \textit{socket} com o mesmo protocolo e família de endereço que o \textit{socket} passado como parâmetro, e também alocar um novo descritor de arquivo para o \textit{socket} criado.
    
    Os parâmetros são:
    \begin{enumerate}
        \item \textit{socket}\\
        Um \textit{socket} criado com a função {\tt socket()}, que foi associado a um endereço com a função {\tt bind()} e que foi usado na função {\tt listen()}.
        \item \textit{address}\\
        Apontador para uma estrutura \textbf{sockaddr} em que o endereço do \textit{socket} usado na conexão deve ser retornado. Pode ser NULL.
        \item \textit{address\_len}\\
        Apontador para um objeto \textbf{socklen\_t} que, na entrada, especifica o tamanho da estrutura \textbf{sockaddr} especificada e, na saída, especifica o tamanho do endereço armazendo. Pode ser NULL se \textit{address} for NULL.
    \end{enumerate}
\item {\tt htons}\\
    Esta função está definida <arpa/inet.h> (ou em <netinet/in.h>) e sua assinatura é:
    \begin{lstlisting}
    uint16_t htons(uint16_t hostshort);
    \end{lstlisting}
\end{enumerate}

\section{Questão 2}

\section{Questão 3}

\section{Questão 4}

\section{Questão 5}

\end{document}

