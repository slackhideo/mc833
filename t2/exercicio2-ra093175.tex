\documentclass[a4paper,10pt]{article}
\usepackage[utf8]{inputenc}
\usepackage[top=2cm, left = 2cm , right=2cm , bottom=2cm]{geometry}
\usepackage{amsmath}
\usepackage{amsfonts}
\usepackage{amssymb}
\usepackage{graphicx}
\usepackage{float}
\usepackage{subcaption}
\usepackage[brazil]{babel}
%\pagestyle{plain}
\usepackage{listings}
\usepackage{color}
\usepackage{hyperref}

\definecolor{dkgreen}{rgb}{0,0.6,0}
\definecolor{gray}{rgb}{0.5,0.5,0.5}
\definecolor{mauve}{rgb}{0.58,0,0.82}

\lstset{frame=tb,
  language=bash,
  aboveskip=3mm,
  belowskip=3mm,
  showstringspaces=false,
  columns=flexible,
  basicstyle={\small\ttfamily},
  numbers=none,
  numberstyle=\tiny\color{gray},
  keywordstyle=\color{blue},
  commentstyle=\color{dkgreen},
  stringstyle=\color{mauve},
  breaklines=true,
  breakatwhitespace=true,
  tabsize=3
}

\hypersetup{
    colorlinks=true,
    linkcolor=blue,
    filecolor=magenta,      
    urlcolor=cyan,
}
 
\urlstyle{same}

\begin{document}
%\twocolumn


\title{MC833 A - Programação de redes de computadores\\
Relatório - Tarefa 02}

\author {093175 - Victor Fernando Pompeo Barbosa - \textit{victorfpb@gmail.com}}

%\date{}

\maketitle

\centerline{Prof. Paulo Licio de Geus}
\centerline{IC -- UNICAMP}

\vspace{2cm}
\tableofcontents
    
%%%%%%%%%%%%%%%%%%%%%%%%%%%%%%%%%%%%%%%%%%%%%%%%%%%%%%%%%%%%%%%%%%%%%%%%%%%%%
\newpage
\section{Introdução}
\hspace{14pt}

    Nesta tarefa estudaremos a ferramenta {\tt tcpdump}, utilizada para análise de protocolos e pacotes de rede. A ferramenta permite ao usuário a exibição de pacotes transmitidos por uma interface de rede presente no computador. 

%%%%%%%%%%%%%%%%%%%%%%%%%%%%%%%%%%%%%%%%%%%%%%%%%%%%%%%%%%%%%%%%%%%%%%%%%%%%%

\section{Desenvolvimento}

O exercício se baseia em um cenário no qual um programa transmite um arquivo da máquina \textit{willow} para a máquina \textit{maple} sobre uma conexão TCP. A ferramenta {\tt tcpdump} foi executada no transmissor para registrar os pacotes enviados e os pacotes de reconhecimento recebidos.\\

O arquivo {\tt tcpdump.dat} contém o log de todos os pacotes TCP do cenário descrito e foi obtido a partir de um link fornecido na página da disciplina. O primeiro passo envolveu a conversão do arquivo {\tt tcpdump.dat} para o formato texto, por meio do comando {\tt tcpdump -r tcpdump.dat > outfile.txt}. \\

\section{Questões}
\begin{enumerate}
\item As interfaces nas quais o {\tt tcpdump} pode escutar/capturar dados podem ser obtidas por meio do comando {\tt tcpdump -D --list-interfaces}. A saída obtida no terminal após a execução do comando pode ser verificada a seguir.

\begin{lstlisting}
niko@ubuntu:~/Desktop/mc833/t2$ tcpdump -D --list-interfaces
1.eth0 [Up, Running]
2.any (Pseudo-device that captures on all interfaces) [Up, Running]
3.lo [Up, Running, Loopback]
4.bluetooth-monitor (Bluetooth Linux Monitor)
5.nflog (Linux netfilter log (NFLOG) interface)
6.nfqueue (Linux netfilter queue (NFQUEUE) interface)
7.usbmon1 (USB bus number 1)
8.usbmon2 (USB bus number 2)
\end{lstlisting}

As interfaces incluem as interfaces mostradas pelo comando {\tt ifconfig}, mas não estão limitadas a elas. O comando {\tt tcpdump} também inclui, além delas, interfaces de Bluetooth e USB, por exemplo.


\item Para poder observar o endereço IP dos nós, a leitura do arquivo de log pode ser feita usando o comando {\tt tcpdump -r tcpdump.dat > outfile-ip.txt -n}. Esse comando evita que os endereços sejam traduzidos para nomes.\\

A primeira linha do arquivo de saída está transcrita abaixo. Nela, é possível observar que o endereço IP do nó \textit{willow} é {\tt 128.30.4.222}, enquanto o do nó \textit{maple} é {\tt 128.30.4.223}. 

\begin{lstlisting}
21:34:41.473036 ARP, Request who-has 128.30.4.223 tell 128.30.4.222, length 28
\end{lstlisting}


\item Para poder observar o endereço MAC dos nós, a leitura do arquivo de log pode ser feita usando o comando {\tt tcpdump -t tcpdump.dat > outfile-mac.txt -e}. Esse comando imprime os endereços MAC dos nós em cada linha.\\

As primeiras linhas do arquivo de saída estão transcritas abaixo. Nelas, é possível observar que o endereço MAC do nó \textit{willow} é {\tt 00:16:ea:8e:28:44} e que o endereço MAC do nó \textit{maple} é {\tt 00:16:ea:8d:e5:8a}.

\begin{lstlisting}
21:34:41.473036 00:16:ea:8e:28:44 (oui Unknown) > Broadcast, ethertype ARP (0x0806), length 42: Request who-has maple.csail.mit.edu tell willow.csail.mit.edu, length 28
21:34:41.473505 00:16:ea:8d:e5:8a (oui Unknown) > 00:16:ea:8e:28:44 (oui Unknown), ethertype ARP (0x0806), length 42: Reply maple.csail.mit.edu is-at 00:16:ea:8d:e5:8a (oui Unknown), length 28
21:34:41.473518 00:16:ea:8e:28:44 (oui Unknown) > 00:16:ea:8d:e5:8a (oui Unknown), ethertype IPv4 (0x0800), length 74: willow.csail.mit.edu.39675 > maple.csail.mit.edu.5001: Flags [S], seq 1258159963, win 14600, options [mss 1460,sackOK,TS val 282136473 ecr 0,nop,wscale 7], length 0
\end{lstlisting}


\item As portas TCP usadas pelos nós podem ser obtidas da saída do comando obtida com a flag {tt -n}. Uma linha qualquer da saída está representada a seguir.

\begin{lstlisting}
21:34:41.474232 IP 128.30.4.222.39675 > 128.30.4.223.5001: Flags [.], seq 2921:4369, ack 1, win 115, options [nop,nop,TS val 282136474 ecr 282202089], length 1448
\end{lstlisting}

É possível observar que o nó com endereço {\tt 128.30.4.222} (o nó \textit{willow}) se comunica a partir da porta 39675, enquanto o nó que possui endereço {\tt 128.30.4.223} (o nó \textit{maple}) se comunica a partir da porta 5001.\\

A porta 5001, usada pelo nó \textit{maple}, é uma porta registrada. Portas registradas são atribuídas pela IANA (Internet Assigned Numbers Authority) para uso por um certo protocolo ou aplicação.


\item O número de kilobytes transferido pode ser deduzido do número de sequência do último pacote enviado/recebido.

\begin{lstlisting}
21:34:44.329956 IP maple.csail.mit.edu.5001 > willow.csail.mit.edu.39675: Flags [.], ack 1572890, win 820, options [nop,nop,TS val 282204945 ecr 282139320], length 0
\end{lstlisting}

Isso indica que o nó \textit{maple} recebeu todos os bytes desde o byte \#0 até o byte \#1572889. Dessa maneira, foram transferidos aproximadamente 1,573 MB ou 1\,573 kB durante essa sessão.\\

A duração da sessão pode ser calculada pela diferença dos \textit{timestamps} que marcam o início e fim da sessão.

\begin{lstlisting}
21:34:41.473036 ARP, Request who-has maple.csail.mit.edu tell willow.csail.mit.edu, length 28
\end{lstlisting}
\begin{lstlisting}
21:34:44.339015 IP willow.csail.mit.edu.39675 > maple.csail.mit.edu.5001: Flags [.], ack 2, win 115, options [nop,nop,TS val 282139339 ecr 282204955], length 0
\end{lstlisting}

Portanto, a sessão durou $2,865979$ segundos. A vazão do fluxo TCP entre os nós foi de, aproximadamente, $548\,813$ kB/s.


\item A linha referente ao envio do pacote {\tt 1473:2921} está representada a seguir. Na sequência, podemos observar a linha referente a seu acknowledgement.

\begin{lstlisting}
21:34:41.474225 IP willow.csail.mit.edu.39675 > maple.csail.mit.edu.5001: Flags [.], seq 1473:2921, ack 1, win 115, options [nop,nop,TS val 282136474 ecr 282202089], length 1448
\end{lstlisting}
\begin{lstlisting}
21:34:41.482047 IP maple.csail.mit.edu.5001 > willow.csail.mit.edu.39675: Flags [.], ack 2921, win 159, options [nop,nop,TS val 282202095 ecr 282136474], length 0
\end{lstlisting}

O round-trip time (RTT) entre os dois nós, baseado no pacote 1473:2921, é, portanto, igual a $0,007822$ segundo.

Por outro lado, as linhas referentes ao envio do pacote {\tt 13057:14505} e ao recebimento de seu acknowledgement estão representadas a seguir.

\begin{lstlisting}
21:34:41.474992 IP willow.csail.mit.edu.39675 > maple.csail.mit.edu.5001: Flags [.], seq 13057:14505, ack 1, win 115, options [nop,nop,TS val 282136474 ecr 282202090], length 1448
\end{lstlisting}
\begin{lstlisting}
21:34:41.499373 IP maple.csail.mit.edu.5001 > willow.csail.mit.edu.39675: Flags [.], ack 14505, win 331, options [nop,nop,TS val 282202114 ecr 282136474], length 0
\end{lstlisting}

Nesse caso, o round-trip time (RTT) entre os dos nós foi de $0,024381$ segundo. A diferença nos dois valores dá-se por conta de a velocidade com que os pacotes são enviados/recebidos é influenciada pelo tráfego de rede e o volume de dados cuja transmissão está planejada.


\item O procedimento de \textit{three-way handshake} é utilizado para dar início à conexão e é identificado pela flag {\tt [S]}. Podemos observá-lo na saída representada a seguir.

\begin{lstlisting}
21:34:41.473518 IP willow.csail.mit.edu.39675 > maple.csail.mit.edu.5001: Flags [S], seq 1258159963, win 14600, options [mss 1460,sackOK,TS val 282136473 ecr 0,nop,wscale 7], length 0
21:34:41.474055 IP maple.csail.mit.edu.5001 > willow.csail.mit.edu.39675: Flags [S.], seq 2924083256, ack 1258159964, win 14480, options [mss 1460,sackOK,TS val 282202089 ecr 282136473,nop,wscale 7], length 0
21:34:41.474079 IP willow.csail.mit.edu.39675 > maple.csail.mit.edu.5001: Flags [.], ack 1, win 115, options [nop,nop,TS val 282136474 ecr 282202089], length 0
\end{lstlisting}	

O procedimento de \textit{connection termination} é identificado pela flag {\tt [F]}. Podemos observá-lo na saída representada a seguir.

\begin{lstlisting}
21:34:44.339007 IP maple.csail.mit.edu.5001 > willow.csail.mit.edu.39675: Flags [F.], seq 1, ack 1572890, win 905, options [nop,nop,TS val 282204955 ecr 282139320], length 0
21:34:44.339015 IP willow.csail.mit.edu.39675 > maple.csail.mit.edu.5001: Flags [.], ack 2, win 115, options [nop,nop,TS val 282139339 ecr 282204955], length 0
\end{lstlisting}

\end{enumerate}

\end{document}

