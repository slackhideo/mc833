\documentclass[a4paper,10pt]{article}
\usepackage[top=2cm, left = 2cm , right=2cm , bottom=2cm]{geometry}
\usepackage{amsmath}
\usepackage{amsfonts}
\usepackage[utf8]{inputenc}
\usepackage{amssymb}
\usepackage{graphicx}
\usepackage{float}
\usepackage{subcaption}
\usepackage{enumerate}
\usepackage[brazil]{babel}
%\pagestyle{plain}
\usepackage{listings}
\usepackage{color}
\usepackage{hyperref}
\usepackage{graphicx}

\definecolor{dkgreen}{rgb}{0,0.6,0}
\definecolor{gray}{rgb}{0.5,0.5,0.5}
\definecolor{mauve}{rgb}{0.58,0,0.82}

\lstset{frame=tb,
  language=bash,
  aboveskip=3mm,
  belowskip=3mm,
  showstringspaces=false,
  columns=flexible,
  basicstyle={\small\ttfamily},
  numbers=none,
  numberstyle=\tiny\color{gray},
  keywordstyle=\color{blue},
  commentstyle=\color{dkgreen},
  stringstyle=\color{mauve},
  breaklines=true,
  breakatwhitespace=true,
  tabsize=3,
  literate={á}{{\'a}}1
           {ç}{{\c{c}}}1
           {ü}{{\"u}}1
           {é}{{\'e}}1
}

\lstset{
}

\hypersetup{
    colorlinks=true,
    linkcolor=blue,
    filecolor=magenta,      
    urlcolor=cyan,
}
 
\urlstyle{same}

\begin{document}
%\twocolumn


\title{MC833 A - Programação de redes de computadores\\
Relatório - Tarefa 06}

\author {   093125 - Tiago Martinho de Barros - \textit{tiago.ec09@gmail.com}\\
            093175 - Victor Fernando Pompêo Barbosa - \textit{victorfpb@gmail.com}}

%\date{}

\maketitle

\centerline{Prof. Paulo Lício de Geus}
\centerline{IC -- UNICAMP}

\vspace{2cm}
\tableofcontents
    
%%%%%%%%%%%%%%%%%%%%%%%%%%%%%%%%%%%%%%%%%%%%%%%%%%%%%%%%%%%%%%%%%%%%%%%%%%%%%
\newpage
\section{Introdução}
\hspace{14pt}

    Nesta tarefa aprenderemos sobre sockets UDP. Além disso, serão construídos um servidor de eco e um cliente UDP. 

%%%%%%%%%%%%%%%%%%%%%%%%%%%%%%%%%%%%%%%%%%%%%%%%%%%%%%%%%%%%%%%%%%%%%%%%%%%%%
\section{Questão 1}

Todas as funções e macros estão definidas em $\langle sys/socket.h \rangle$.

\begin{itemize}
\item {\tt recvfrom}

Assinatura:
\begin{lstlisting}
 ssize_t recvfrom(int socket, void *restrict buffer, size_t length,
       int flags, struct sockaddr *restrict address,
       socklen_t *restrict address_len);
\end{lstlisting}

A função {\tt recvfrom} é usada para receber mensagens de um socket, tanto orientados a conexão como não orientados a conexão. No entanto, ela é principalmente utilizada em sockets não orientados a conexão, pois permite à aplicação recuperar o endereço do remetente da mensagem.

O tipo {\tt ssize\_t} representa o tamanho de blocos que podem ser lidos/escritos em uma única operação. É um inteiro com sinal de pelo menos 16 bits.

Seus argumentos são:
\begin{enumerate}[a)]
\item \textit{\underline{socket}}\\
Especifica o socket a ser utilizado para a comunicação.

\item \textit{\underline{buffer}}\\
É um apontador para o buffer que receberá a mensagem.

\item \textit{\underline{length}}\\
Especifica o tamanho (em bytes) do buffer apontado por \textit{\underline{buffer}}. É o tamanho máximo de mensagem que o buffer consegue armazenar.

\item \textit{\underline{flags}}\\
Especifica o tipo de recebimento de mensagem. As opções são {\tt MSG\_PEEK}, {\tt MSG\_OOB} e {\tt MSG\_WAITALL}.

\item \textit{\underline{address}}\\
Pode conter um apontador nulo ou um ponteiro para uma estrutura {\tt sockaddr} onde o endereço do remetente será armazenado.

\item \textit{\underline{address\_len}}\\
Especifica o tamanho da estrutura {\tt sockaddr} apontada pelo argumento \textit{\underline{address}}.
\end{enumerate}

O valor de retorno é o tamanho da mensagem (em bytes). Se nenhuma mensagem estiver disponível, mas a conexão for encerrada corretamente, o valor de retorno é zero. Caso contrário, a função retorna -1 e {\tt errno} é setado para indicar o tipo de erro.

\item {\tt sendto}

Assinatura:
\begin{lstlisting}
 ssize_t sendto(int socket, const void *message, size_t length,
       int flags, const struct sockaddr *dest_addr,
       socklen_t dest_len);\end{lstlisting}

A função {\tt sendto} é usada para enviar mensagens por meio de um socket, tanto orientados a conexão como não orientados a conexão. Caso o socket não seja orientado a conexão, a mensagem será enviada ao endereço especificado em \textit{\underline{dest\_addr}}; caso ele seja, esse argumento é ignorado.

Seus argumentos são:
\begin{enumerate}[a)]
\item \textit{\underline{socket}}\\
Especifica o socket a ser utilizado para a comunicação.

\item \textit{\underline{message}}\\
É um apontador para o buffer contendo a mensagem a ser enviada.

\item \textit{\underline{length}}\\
Especifica o tamanho da mensagem.

\item \textit{\underline{flags}}\\
Especifica o tipo de recebimento de mensagem. As opções são {\tt MSG\_EOR} e {\tt MSG\_OOB}.

\item \textit{\underline{dest\_addr}}\\
É um ponteiro para uma estrutura {\tt sockaddr} onde o endereço de destino está armazenado.

\item \textit{\underline{dest\_len}}\\
Especifica o tamanho da estrutura {\tt sockaddr} apontada pelo argumento \textit{\underline{dest\_addr}}.
\end{enumerate}

O valor de retorno é o tamanho da mensagem enviada (em bytes). Caso contrário, a função retorna -1 e {\tt errno} é setado para indicar o tipo de erro.

\end{itemize}


\section{Questão 2}
Testamos o servidor \textit{echo\_server\_select\_tcp} com nossos clientes das tarefas 3 e 4 (alterando o valor da porta de conexão), com o telnet e com o netcat.
Este servidor não produz nenhuma saída no terminal que o executa, apenas ecoa para o cliente o texto que lhe foi enviado.
Abaixo temos as saídas dos clientes:

\begin{itemize}
\item Cliente da tarefa 3

\begin{lstlisting}
bash-4.3$ ./client-t3 localhost
Cliente T3
Cliente T3

\end{lstlisting}

\item Cliente da tarefa 4

\begin{lstlisting}
bash-4.3$ ./client-t4 localhost
-------------------
IP local: 127.0.0.1
Porta local: 52809
-------------------

Cliente T4
Cliente T4
\end{lstlisting}

\item Cliente telnet

\begin{lstlisting}
bash-4.3$ telnet localhost 56789
Trying 127.0.0.1...
Connected to localhost.
Escape character is '^]'.
Cliente telnet
Cliente telnet

\end{lstlisting}

\item Cliente netcat

\begin{lstlisting}
bash-4.3$ nc localhost 56789
Cliente netcat
Cliente netcat

\end{lstlisting}

\end{itemize}

Todos esses clientes foram executados "simultaneamente", como mostra o {\tt netstat} (omitindo as linhas de conexões irrelevantes para o nosso caso):

\begin{lstlisting}
bash-4.3$ netstat -tu
Active Internet connections (w/o servers)
Proto Recv-Q Send-Q Local Address           Foreign Address         State      
tcp        0      0 localhost.localdo:56789 localhost.localdo:52748 ESTABLISHED
tcp        0      0 localhost.localdo:52809 localhost.localdo:56789 ESTABLISHED
tcp        0      0 localhost.localdo:56789 localhost.localdo:52809 ESTABLISHED
tcp        0      0 localhost.localdo:56789 localhost.localdo:52817 ESTABLISHED
tcp        0      0 localhost.localdo:52748 localhost.localdo:56789 ESTABLISHED
tcp        0      0 localhost.localdo:52915 localhost.localdo:56789 ESTABLISHED
tcp        0      0 localhost.localdo:56789 localhost.localdo:52915 ESTABLISHED
tcp        0      0 localhost.localdo:52817 localhost.localdo:56789 ESTABLISHED

\end{lstlisting}

Notamos os pares de conexões \{localhost:56789 $\leftrightarrow$ localhost:52748, localhost:56789 $\leftrightarrow$ localhost:52809, localhost:56789 $\leftrightarrow$ localhost:52817 e localhost:56789 $\leftrightarrow$ localhost:52915\}.

\section{Questão 3}
O servidor funciona bloqueando na função {\tt select} esperando que, pelo menos, um dos descritores de arquivos do conjunto \textit{rset} tenha conteúdo para ser lido. Neste conjunto estão o descritor \textit{listenfd} (que escuta novas conexões) e um \textit{connfd} para cada conexão ativa (que escuta cada cliente).

Assim, quando uma nova requisição de conexão chegar (nova atividade no \textit{listenfd}), o servidor vai aceitar, armazenar o recém-gerado descritor de arquivos do socket aceito e incluir esse descritor de arquivos no conjunto de todos os descritores (\textit{allset}), cuja cópia (\textit{rset}) será usada na chamada da função {\tt select}. E depois chama a função {\tt select}, que irá verificar nova atividade nos descritores de arquivos. Se houver, age de acordo; se não houver, espera até que haja atividade.

Quando um cliente com uma conexão com o servidor já estabelecida envia texto para o servidor (nova atividade em um dos \textit{connfd}), a mesma chamada à função {\tt select} vai detectar isso e o código verifica se foi o \textit{listenfd} que recebeu essa atividade. Como nesse caso, não é, o servidor procura quais dos descritores de arquivos (quais clientes) receberam atividade (usando FD\_ISSET), lê o texto enviado por cada cliente e os envia de volta para o respectivo cliente. Depois, chama a função {\tt select} para processar a próxima atividade (nova conexão ou texto enviado por cliente) se houver, ou esperar uma nova atividade em um dos descritores de arquivos.

Este servidor não é paralelo realmente, mas como ele não bloqueia na função \textit{read} (pois só a chama quando houver texto a ser lido), pode atender a vários clientes "simultaneamente", pois as requisições de conexões dos clientes são enfileiradas e tratadas uma a uma; e os textos enviados por cada cliente estão associados ao seu respectivo \textit{socket} (descritor de arquivos) e também são tratados um a um. Desta forma, o servidor não executa nada em paralelo, porém como seu processamento é rápido, ele consegue atender mais de um usuário "simultaneamente".

\section{Questão 4}

No código fornecido do programa \textit{echo\_server\_select\_tcp}, algumas funções de uso chave para o funcionamento do servidor não tinham seus valores de retorno checados, possibilitando que o programa assumisse comportamento inesperado em caso de erro. São elas: {\tt select, accept, read} e {\tt send}. Dessa maneira, cada uma delas passou a ter seu valor de retorno checado e, caso o valor retornado indique erro, o servidor imprime na tela uma mensagem especificando a função problemática. No caso das funções {\tt select} e {\tt accept}, o programa é terminado.

Como exemplo, utilizaremos a função {\tt select}. A chamada exibida a seguir estava presente no programa original.

\begin{lstlisting}
		nready = select(maxfd+1, &rset, NULL, NULL, NULL);
\end{lstlisting} 

Com a adição da checagem dos valores de retorno, o trecho de código a seguir substituiu o anterior.

\begin{lstlisting}
		if( (nready = select(maxfd+1, &rset, NULL, NULL, NULL)) == -1 ){
			perror("simplex-talk: select");
			exit(1);
		}

\end{lstlisting}

As outras funções sofreram modificações semelhantes.


\section{Questão 5}

O uso de {\tt fork} é uma forma de lidar com múltiplas conexões, adotando como saída a utilização de múltiplos processos. As principais desvantagens dessa abordagem são duas:
\begin{enumerate}
\item Caso o servidor tenha que lidar com um grande número de conexões simultâneas, como cada uma delas utiliza o {\tt fork}, o servidor pode exaurir a memória de aplicação, alcançando o número máximo de processos;
\item Como cada chamada de {\tt fork} duplica variáveis, descritores de arquivos e todo o contexto do programa, o servidor pode ficar sem memória no caso de cada conexão necessitar de qualquer tipo de processamento significante.
\end{enumerate}

Por outro lado, o uso de {\tt select} gerencia múltiplas conexões utilizando apenas um processo. Dessa maneira, os problemas referentes ao overhead de duplicar processos não existem. Essa saída é mais custosa para o programador, por ser mais complexa, mas lida bem com os problemas inerentes ao uso de {\tt fork}. Contudo, o uso do {\tt select} tem a limitação do \textbf{FD\_SETSIZE}, que define o tamanho do conjunto de descritores de arquivos e, na máquina utilizada, é 1024. Ou seja, a aplicação consegue atender a 1024 clientes simultaneamente, enquanto que a solução com {\tt fork} está limitada pelo número de processos que um usuário pode ter executando que, na máquina utilizada, é 31161 (obtido via {\tt ulimit -u}).

O uso de {\tt select} também enfrenta problemas de escalabilidade advindos do funcionamento da função. Quando há um número grande de clientes conectados simultaneamente, o custo de criar, manter e checar o conjunto de descritores de arquivos é bastante grande. 

No entanto, a chamada à função {\tt select} cria uma situação de concorrência \textit{aparente}, com a habilidade de lidar com múltiplas conexões, mas sem o uso de diferentes threads ou processos. Na prática, isso significa que o servidor gerencia as diferentes conexões \textit{sem troca de contexto}. Dessa maneira, é possível compartilhar dados entre as diferentes sessões e clientes. Caso isso seja desejável, isso é uma vantagem do uso de {\tt select}; no entanto, exige um cuidado adicional do programador para que dados não sejam erroneamente compartilhados entre clientes distintos, causando problemas de segurança.

\section{Questão 6}


\section{Questão 7}


\section{Questão 8}

\end{document}

