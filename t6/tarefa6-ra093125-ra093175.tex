\documentclass[a4paper,10pt]{article}
\usepackage[top=2cm, left = 2cm , right=2cm , bottom=2cm]{geometry}
\usepackage{amsmath}
\usepackage{amsfonts}
\usepackage[utf8]{inputenc}
\usepackage{amssymb}
\usepackage{graphicx}
\usepackage{float}
\usepackage{subcaption}
\usepackage{enumerate}
\usepackage[brazil]{babel}
%\pagestyle{plain}
\usepackage{listings}
\usepackage{color}
\usepackage{hyperref}
\usepackage{graphicx}

\definecolor{dkgreen}{rgb}{0,0.6,0}
\definecolor{gray}{rgb}{0.5,0.5,0.5}
\definecolor{mauve}{rgb}{0.58,0,0.82}

\lstset{frame=tb,
  language=bash,
  aboveskip=3mm,
  belowskip=3mm,
  showstringspaces=false,
  columns=flexible,
  basicstyle={\small\ttfamily},
  numbers=none,
  numberstyle=\tiny\color{gray},
  keywordstyle=\color{blue},
  commentstyle=\color{dkgreen},
  stringstyle=\color{mauve},
  breaklines=true,
  breakatwhitespace=true,
  tabsize=3,
  literate={á}{{\'a}}1
           {ç}{{\c{c}}}1
           {ü}{{\"u}}1
           {é}{{\'e}}1
}

\lstset{
}

\hypersetup{
    colorlinks=true,
    linkcolor=blue,
    filecolor=magenta,      
    urlcolor=cyan,
}
 
\urlstyle{same}

\begin{document}
%\twocolumn


\title{MC833 A - Programação de redes de computadores\\
Relatório - Tarefa 06}

\author {   093125 - Tiago Martinho de Barros - \textit{tiago.ec09@gmail.com}\\
            093175 - Victor Fernando Pompêo Barbosa - \textit{victorfpb@gmail.com}}

%\date{}

\maketitle

\centerline{Prof. Paulo Lício de Geus}
\centerline{IC -- UNICAMP}

\vspace{2cm}
\tableofcontents
    
%%%%%%%%%%%%%%%%%%%%%%%%%%%%%%%%%%%%%%%%%%%%%%%%%%%%%%%%%%%%%%%%%%%%%%%%%%%%%
\newpage
\section*{Introdução}
\hspace{14pt}

    Nesta tarefa aprenderemos sobre sockets UDP. Além disso, serão construídos um servidor de eco e um cliente UDP. 

%%%%%%%%%%%%%%%%%%%%%%%%%%%%%%%%%%%%%%%%%%%%%%%%%%%%%%%%%%%%%%%%%%%%%%%%%%%%%
\renewcommand\thepart{\Alph{part}}
\part{Cliente-Servidor de eco com sockets UDP}
\section{Questão 1}

Todas as funções e macros estão definidas em $\langle sys/socket.h \rangle$.

\begin{itemize}
\item {\tt recvfrom}

Assinatura:
\begin{lstlisting}
 ssize_t recvfrom(int socket, void *restrict buffer, size_t length,
       int flags, struct sockaddr *restrict address,
       socklen_t *restrict address_len);
\end{lstlisting}

A função {\tt recvfrom} é usada para receber mensagens de um socket, tanto orientados a conexão como não orientados a conexão. No entanto, ela é principalmente utilizada em sockets não orientados a conexão, pois permite à aplicação recuperar o endereço do remetente da mensagem.

O tipo {\tt ssize\_t} representa o tamanho de blocos que podem ser lidos/escritos em uma única operação. É um inteiro com sinal de pelo menos 16 bits.

Seus argumentos são:
\begin{enumerate}[a)]
\item \textit{\underline{socket}}\\
Especifica o socket a ser utilizado para a comunicação.

\item \textit{\underline{buffer}}\\
É um apontador para o buffer que receberá a mensagem.

\item \textit{\underline{length}}\\
Especifica o tamanho (em bytes) do buffer apontado por \textit{\underline{buffer}}. É o tamanho máximo de mensagem que o buffer consegue armazenar.

\item \textit{\underline{flags}}\\
Especifica o tipo de recebimento de mensagem. As opções são {\tt MSG\_PEEK}, {\tt MSG\_OOB} e {\tt MSG\_WAITALL}.

\item \textit{\underline{address}}\\
Pode conter um apontador nulo ou um ponteiro para uma estrutura {\tt sockaddr} onde o endereço do remetente será armazenado.

\item \textit{\underline{address\_len}}\\
Especifica o tamanho da estrutura {\tt sockaddr} apontada pelo argumento \textit{\underline{address}}.
\end{enumerate}

O valor de retorno é o tamanho da mensagem (em bytes). Se nenhuma mensagem estiver disponível, mas a conexão for encerrada corretamente, o valor de retorno é zero. Caso contrário, a função retorna -1 e {\tt errno} é setado para indicar o tipo de erro.

\item {\tt sendto}

Assinatura:
\begin{lstlisting}
 ssize_t sendto(int socket, const void *message, size_t length,
       int flags, const struct sockaddr *dest_addr,
       socklen_t dest_len);\end{lstlisting}

A função {\tt sendto} é usada para enviar mensagens por meio de um socket, tanto orientados a conexão como não orientados a conexão. Caso o socket não seja orientado a conexão, a mensagem será enviada ao endereço especificado em \textit{\underline{dest\_addr}}; caso ele seja, esse argumento é ignorado.

Seus argumentos são:
\begin{enumerate}[a)]
\item \textit{\underline{socket}}\\
Especifica o socket a ser utilizado para a comunicação.

\item \textit{\underline{message}}\\
É um apontador para o buffer contendo a mensagem a ser enviada.

\item \textit{\underline{length}}\\
Especifica o tamanho da mensagem.

\item \textit{\underline{flags}}\\
Especifica o tipo de recebimento de mensagem. As opções são {\tt MSG\_EOR} e {\tt MSG\_OOB}.

\item \textit{\underline{dest\_addr}}\\
É um ponteiro para uma estrutura {\tt sockaddr} onde o endereço de destino está armazenado.

\item \textit{\underline{dest\_len}}\\
Especifica o tamanho da estrutura {\tt sockaddr} apontada pelo argumento \textit{\underline{dest\_addr}}.
\end{enumerate}

O valor de retorno é o tamanho da mensagem enviada (em bytes). Caso contrário, a função retorna -1 e {\tt errno} é setado para indicar o tipo de erro.

\end{itemize}


\section{Questão 2}

Foram implementados um cliente e um servidor de eco usando o protocolo UDP. Um teste usando o binário do cliente é mostrado abaixo:

\begin{itemize}
\item Cliente UDP 1

\begin{lstlisting}
bash-4.3$ ./client_udp.out localhost
Oi do cliente 1
Oi do cliente 1

\end{lstlisting}

\item Cliente UDP 2

\begin{lstlisting}
bash-4.3$ ./client_udp.out localhost
Oi do cliente 2
Oi do cliente 2

\end{lstlisting}

\item Cliente netcat

\begin{lstlisting}
bash-4.3$ nc -u localhost 31472
Oi do netcat   
Oi do netcat

\end{lstlisting}


\item Servidor UDP

\begin{lstlisting}
bash-4.3$ ./server_udp.out
Oi do cliente 1

Oi do cliente 2

Oi do netcat   

\end{lstlisting}
\end{itemize}

Assim, verificamos que o servidor suporta atendimento a vários clientes ``simultaneamente''.

\section{Questão 3}


\section{Questão 4}
(rascunho)

\begin{lstlisting}
bash-4.3$ netstat -lnpu

nc -u localhost 44703
\end{lstlisting}




\section{Questão 5}


\part{Erros Assíncronos e Sockets UDP Conectados}
\section{Questão 6}


\section{Questão 7}


\section{Questão 8}

\end{document}

