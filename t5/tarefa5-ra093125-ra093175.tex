\documentclass[a4paper,10pt]{article}
\usepackage[top=2cm, left = 2cm , right=2cm , bottom=2cm]{geometry}
\usepackage{amsmath}
\usepackage{amsfonts}
\usepackage[utf8]{inputenc}
\usepackage{amssymb}
\usepackage{graphicx}
\usepackage{float}
\usepackage{subcaption}
\usepackage[brazil]{babel}
%\pagestyle{plain}
\usepackage{listings}
\usepackage{color}
\usepackage{hyperref}
\usepackage{graphicx}

\definecolor{dkgreen}{rgb}{0,0.6,0}
\definecolor{gray}{rgb}{0.5,0.5,0.5}
\definecolor{mauve}{rgb}{0.58,0,0.82}

\lstset{frame=tb,
  language=bash,
  aboveskip=3mm,
  belowskip=3mm,
  showstringspaces=false,
  columns=flexible,
  basicstyle={\small\ttfamily},
  numbers=none,
  numberstyle=\tiny\color{gray},
  keywordstyle=\color{blue},
  commentstyle=\color{dkgreen},
  stringstyle=\color{mauve},
  breaklines=true,
  breakatwhitespace=true,
  tabsize=3,
  literate={á}{{\'a}}1
           {ç}{{\c{c}}}1
           {ü}{{\"u}}1
           {é}{{\'e}}1
}

\lstset{
}

\hypersetup{
    colorlinks=true,
    linkcolor=blue,
    filecolor=magenta,      
    urlcolor=cyan,
}
 
\urlstyle{same}

\begin{document}
%\twocolumn


\title{MC833 A - Programação de redes de computadores\\
Relatório - Tarefa 05}

\author {   093125 - Tiago Martinho de Barros - \textit{tiago.ec09@gmail.com}\\
            093175 - Victor Fernando Pompêo Barbosa - \textit{victorfpb@gmail.com}}

%\date{}

\maketitle

\centerline{Prof. Paulo Lício de Geus}
\centerline{IC -- UNICAMP}

\vspace{2cm}
\tableofcontents
    
%%%%%%%%%%%%%%%%%%%%%%%%%%%%%%%%%%%%%%%%%%%%%%%%%%%%%%%%%%%%%%%%%%%%%%%%%%%%%
\newpage
\section{Introdução}
\hspace{14pt}

    Nesta tarefa estudaremos e melhoraremos um servidor TCP concorrente que usa a função {\tt select}.

%%%%%%%%%%%%%%%%%%%%%%%%%%%%%%%%%%%%%%%%%%%%%%%%%%%%%%%%%%%%%%%%%%%%%%%%%%%%%
\section{Questão 1}
O código do programa \textit{echo\_server\_select\_tcp} foi estudado e abaixo se encontram explicações sobre algumas funções e macros que são usadas no programa.

Todas as funções e macros estão definidas em $\langle sys/select.h \rangle$.

\begin{itemize}
\item select

Assinatura:
\begin{lstlisting}
int select(int nfds, fd_set *readfds, fd_set *writefds, fd_set *exceptfds, struct timeval *timeout);
\end{lstlisting}
Esta função bloqueia o processo que a invoca até que haja atividade em qualquer um dos conjuntos de descritores de arquivos especificados ou até que um certo tempo passe.

\underline{nfds} é o descritor de arquivos com o maior número dentre os descritores dos conjuntos passados à função (somado com 1). O conjunto de descritores de arquivos \underline{readfds} são verificados quanto à disponibilidade de leitura; o conjunto de descritores de arquivos \underline{writefds} são verificados quanto à disponibilidade de escrita; e o conjunto de descritores de arquivos \underline{exceptfds} são verificados quanto a condições excepcionais. Quando a função é executada, esses conjuntos são modificados para indicar quais descritores de arquivos mudaram de condição. E \underline{timeout} especifica o tempo máximo para se esperar.

\item FD\_ZERO

Assinatura:
\begin{lstlisting}
void FD_ZERO(fd_set *set);
\end{lstlisting}
Esta macro inicializa o conjunto de descritores de arquivos \underline{set} como um conjunto vazio.

\item FD\_SET

Assinatura:
\begin{lstlisting}
void FD_SET(int fd, fd_set *set);
\end{lstlisting}
Esta macro adiciona o descritor de arquivos \underline{fd} ao conjunto de descritores de arquivos \underline{set}.

\item FD\_ISSET

Assinatura:
\begin{lstlisting}
int FD_ISSET(int fd, fd_set *set);
\end{lstlisting}
Esta macro verifica se o descritor de arquivos \underline{fd} faz parte do conjunto de descritores de arquivos \underline{set}. Se sim, retorna um valor não nulo (true); caso contrário, retorna 0 (false).

\item FD\_CLR

Assinatura:
\begin{lstlisting}
void FD_CLR(int fd, fd_set *set);
\end{lstlisting}
Esta macro remove o descritor de arquivos \underline{fd} do conjunto de descritores de arquivos \underline{set}.
\end{itemize}

\section{Questão 2}

\section{Questão 3}

\section{Questão 4}

\section{Questão 5}

\end{document}

